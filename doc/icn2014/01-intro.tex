\section{Introduction}
At its core, the design and architecture of today's Internet is communication-based packet switching network. Since its inception, it has been retrofitted with a variety of transport and application layer protocols and middleware to support a growing set of consumer applications, such as the Web, email, and perhaps most importantly in recent times, media streaming services. The latter type of applications are bandwidth-intensive content distribution applications which leverage the underlying communication-based network as a distribution network, leading to a massive consumption of vital networking resources. 

Information-centric networks (ICNs) are a new class of network architecture designs that aim to address this increasingly popular type of network traffic by decoupling data from its source and shifting the emphasis of addressable content from hosts and interfaces to content \cite{first}. By directly addressing content instead of hosts, content dissemination and security can be ``distributed'' throughout the network in the sense that consumers requests for content are satisifed by \emph{any} resource in the network (i.e., not necessarily the original producer). For example, routers close to consumers may cache content with a particular name and then satisfy all content requests that match the content's name. In-network caching and data-centric security measures are two of the defining characteristics of these new network designs. 

As of today there are several information-centric networking proposals being explored as alternative designs to today's Internet; Named Data Networking (NDN) \cite{ndn-techreport} is one of the more promising designs that is still an active area of active research (see \url{www.named-data.net} for more information). As a replacement for IP-based networks, the complete adoption of NDN, or any one of these designs, will realistically need to be done by slow and continual integration and replacement of IP-based networking resources with NDN-based resources. Currently, however, there is no engineering plan to support the IP-to-ICN integration without significant software modification and application, transport, or network layer source code modifications.

Consequently, the primary objective of this work is to aid the integration of future content-centric networking resources into the existing IP-centric Internet by providing an application and transport layer gateway between IP and NDN resources. We assume that NDN \emph{will not} be deployed over the IP network, as such a deployment scheme would nullify the need for a gateway between the two networks. Application-layer traffic corresponding to protocols such as HTTP, FTP, SMTP, IMAP, etc. will be translated by middleware running in such gateways to correctly interface with the NDN resources, thereby serving as a semantic bridge between these two fundamentally different networking architectures. Similarly, using a custom NDN-to-IP protocol, NDN \emph{interests} for content will be translated to the messages adhereing to the corresponding application-layer protocol to traverse the IP network. When serving as a bridge between islands of NDN resources, two gateways will leverage the features of the IP network to move messages among two separate NDN beds. Specifically, NDN interests may be encapsulated in UDP datagrams composed of IP packets traversing an IP network between two NDN islands. The primary benefit of this semi-transparent gateway is that existing IP-centric applications need not be modified at any layer in the network stack to interoperate with NDN resources. 

\todo[inline]{Emphasize notlike a broker}

% \todo[inline]{we assume NDN is NOT deployed over IP-i.e., computers may speak IP or NDN, but not both}

% The rest of this paper is outlined as follows***


