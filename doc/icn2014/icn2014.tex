\documentclass{sigcomm}

\usepackage{todonotes}
\usepackage{algorithm}
\usepackage{url}
\usepackage[noend]{algpseudocode}
\usepackage{mdframed}
\usepackage{amsmath}
\usepackage{paralist}
\input{bnf}

\newtheorem{defn}{\textbf{Definition}}
\newtheorem{thm}{\textbf{Theorem}}
\newtheorem{cor}{\textbf{Corollary}}
\newtheorem{lemma}{\textbf{Lemma}}

% \newcommand{\AND}{{\sf AC$^{\mbox{{\small 3}}}$N}}
\newcommand{\sink}{{\sf CCNSink}}

\begin{document}

% ccn gateway and bridge
\title{CCNSink: An Application-Layer Gateway for TCP/IP and Content Centric Network Interoperability}

\numberofauthors{1} 
\author{
\alignauthor
Christopher A. Wood\\ %\titlenote{NSF GRFP blurb.}
       \affaddr{University of California Irvine}\\
       \affaddr{Irvine, CA, USA}\\
       \email{woodc1@uci.edu}
}

\date{\today}

\maketitle
\begin{abstract}
With the growing presence of data streaming services and applications in today's Internet, content-centric networks (CCNs) are an increasingly attractive design alternative to the traditional IP-based host-oriented architecture. CCNs emphasize content by making it directly addressable and routable within the network, in contrast to addressable hosts and interfaces. This fundamental difference leads to vastly different mechanisms to publish and retrieve content and enable peer-to-peer communication. Named Data Networking (NDN) and its sibling implementation - CCNx - is one particular CCN design that has received considerable academic and industry attention. Despite the many promising benefits, there has been little research into the NDN deployment strategy. Clearly, an incremental deployment strategy is the only viable solution. During this integration phase, however, there will undoubtedly be a need for IP-based (NDN-based) hosts to communicate with and retrieve content from NDN-based (IP-based) hosts. To this end, we present an IP/NDN gateway to support interoperability between these different networking architectures with minimal application and transport layer modifications via semantic translation between the communication mechanisms used in both networks. The performance overhead induced by this gateway is studied in unidirectional and bidirectional communication settings.
\end{abstract}

\category{H.4}{TODO}{TODO}
\category{D.2.8}{TODO}{TODO}[TODO]

\terms{NDN/CCN; CCN Deployment; Network Gateway}

%\keywords{TODO} % NOT required for Proceedings

\section{Introduction}
At its core, the design and architecture of today's Internet is communication-based packet switching network. Since its inception, it has been retrofitted with a variety of transport and application layer protocols and middleware to support a growing set of consumer applications, such as the Web, email, and perhaps most importantly in recent times, media streaming services. The latter type of applications are bandwidth-intensive content distribution applications which leverage the underlying communication-based network as a distribution network, leading to a massive consumption of vital networking resources. 

Information-centric networks (ICNs) are a new class of network architecture designs that aim to address this increasingly popular type of network traffic by decoupling data from its source and shifting the emphasis of addressable content from hosts and interfaces to content \cite{first}. By directly addressing content instead of hosts, content dissemination and security can be ``distributed'' throughout the network in the sense that consumers requests for content are satisifed by \emph{any} resource in the network (i.e., not necessarily the original producer). For example, routers close to consumers may cache content with a particular name and then satisfy all content requests that match the content's name. In-network caching and data-centric security measures are two of the defining characteristics of these new network designs. 

As of today there are several information-centric networking proposals being explored as alternative designs to today's Internet; Named Data Networking (NDN) \cite{ndn-techreport} is one of the more promising designs that is still an active area of active research (see \url{www.named-data.net} for more information). As a replacement for IP-based networks, the complete adoption of NDN, or any one of these designs, will realistically need to be done by slow and continual integration and replacement of IP-based networking resources with NDN-based resources. Currently, however, there is no engineering plan to support the IP-to-ICN integration without significant software modification and application, transport, or network layer source code modifications.

Consequently, the primary objective of this work is to aid the integration of future content-centric networking resources into the existing IP-centric Internet by providing an application and transport layer gateway between IP and NDN resources. We assume that NDN \emph{will not} be deployed over the IP network, as such a deployment scheme would nullify the need for a gateway between the two networks. Application-layer traffic corresponding to protocols such as HTTP, FTP, SMTP, IMAP, etc. will be translated by middleware running in such gateways to correctly interface with the NDN resources, thereby serving as a semantic bridge between these two fundamentally different networking architectures. Similarly, using a custom NDN-to-IP protocol, NDN \emph{interests} for content will be translated to the messages adhereing to the corresponding application-layer protocol to traverse the IP network. When serving as a bridge between islands of NDN resources, two gateways will leverage the features of the IP network to move messages among two separate NDN beds. Specifically, NDN interests may be encapsulated in UDP datagrams composed of IP packets traversing an IP network between two NDN islands. The primary benefit of this semi-transparent gateway is that existing IP-centric applications need not be modified at any layer in the network stack to interoperate with NDN resources. 

\todo[inline]{Emphasize notlike a broker}

% \todo[inline]{we assume NDN is NOT deployed over IP-i.e., computers may speak IP or NDN, but not both}

% The rest of this paper is outlined as follows***



\section{NDN/CCNx Overview}
NDN is one of the three currently active NSF FIA (Future Internet Architecture) projects \cite{ndn-techreport}. CCNx is a closely related and more commercially-oriented design backed by PARC \cite{ccnx}\footnote{Due to the similarity between NDN and CCN with respect to this work, we use the terms NDN and CCN interchangeably}. The defining characteristic of both designs is that they decouple location of content from its original publisher. To obtain content, a consumer issues a request (called an {\em interest}) referencing the name of the desired content. An interest is routed based on the specified name, rather than a destination address, over a sequence of routers, each of which keeps state of the forwarded interest. Requested content might be found either in a router that cached it based on a prior interest, or at the producer. Regardless, each router is expected (though not mandated) to cache each content it forwards. In essence, router caches and addressable content enable NDN/CCNx to reduce congestion and latency by keeping content closer to consumers. 

An \emph{interest}, though intended to carry a meaningful (human-readable) URI-like name, can in fact carry arbitrary strings corresponding to any type of data, such as encoded binary. Upon receiving an interest, a router looks up the content by name in its local cache, deemed the \emph{content store} (CS). A match in the CS causes the associated content to be forwarded downstream over the same interface upon which the interest arrived. Interests that do not match any cached content are stored in a \emph{Pending Interest Table} (PIT) together with the incoming and the outgoing interfaces. The interest is forwarded based on the longest-prefix match in the local \emph{Forwarding Information Base} (FIB) table. Multiple interests matching the same name are collapsed into a single PIT entry to prevent redundant interests being sent upstream. Once a content matching a PIT entry is received by a router, it is cached (unless the interest explicitly asks not to cache this content) and forwarded to all incoming interfaces associated with the PIT entry. Finally, the PIT entry is flushed.

One may perceive the pull-based model of content retrieval as an analogue of traditional publish-subscribe distributed systems \cite{eugster2003many}. In fact, the two designs are similar in the sense that interested parties subscribe to content and then receive new content when it becomes available. However, the key difference between these two systems is that NDN (and ICNs in general) ``subscriptions'' are ephemeral. A \emph{single} interest, which is analogous to a subscription, is issued for a \emph{single}, \emph{discrete} piece of content. In contrast to publish-subscribe systems, which push new content to all subscribed users when it becomes available, new NDN content will not be sent to consumers unless they issue another explicit interest in such content. Clearly, publish-subscribe systems can be used to implement NDN. However, this strategy would be extremely wasteful in terms of both computation and storage. Firstly, subscriptions would be continually added and removed for every interest, incuding a lot of computational overhead. Secondly, routing information is distributed in NDN routers via routing tables, whereas routing information would be isolated in clients or a global directory service used to locate content publishers. 

The interest-based strategy for content naming may also be contrived as a close sibling to MIT's Intentional Naming System (INS) \cite{INS}. In fact, the naming schemes and routing strategies are quite different. The INS uses name specifiers composed of attributes (keys) and values, e.g., {\tt [city = washington [building = whitehouse]] [[service = camera [data-type = picture]]] [name = president.jpg]}. The INS resolution service parses each component of these name specifiers to locate the source of the specified data (in this case, where the picture named ``president.jpg'' can be located). In NDN, such a picture might be retrieved by issuing an interest with the name {\tt ccnx:/washington/whitebouse/service/camera/picture/name/president.jpg}. In addition to these explicit naming differences, it is important to notice that INS uses distributed \emph{interest name resolvers} (INRs), similar to hierarchichal DNS servers, to direct name specifiers to content providers, whereas NDN interests are iteratively resolved by routers in the network using their internal routing tables. Active INRs, which effectively serve as ``routers'' of name specifiers, are managed and discovered using a DNS-like service, whereas actual NDN routers are discovered using standard routing protocols (e.g., OSPF, RIP, etc). Clearly, these subtle differences make NDN and INS differentiate to a significant degree. However, INS is in some sense more advanced than NDN in that it also provides a name discovery service. 

Beyond the pull-based content retrieval model that guarantees symmetric interest and content flow, content-centric traffic in NDN and CCNx has strong security implications. Most importantly, security is coupled to content rather than the channel of distribution. All sensitive content must therefore be encrypted in a meaningful way so as to ensure confidentiality. Content integrity and origin authenticity are ensured by mandating that all content be digitally signed by its producers. As we will discuss in the following sections, this requirement plays a crucial role in the bridge component of \sink. Contrary to generating digital signatures, routers need not verify signatures as content flows through the network due to the obvious computational overhead. On the contrary, consumers are assumed to verify all content signatures and re-issue interests in the event that signature verification fails. Issues regarding signature verification and public key distribution are elaborated upon in \cite{ghali2014elements}. 


\section{Heterogeneous Networks and the Gateway Bridge}
Consider the typical hourglass network stack in IP-based networks as shown in left-hand image of Figure \ref{fig:hourglass}. This layered design with a thin-waist infrastructure (IP packets for traffic flow) is what enabled the Internet to grow and expand at such a rapid rate; higher layers in the protocol stack extend this communication medium with support for a variety of applications and networking features (e.g., reliable message traversal via TCP). While the NDN architecture introduces a fundamental paradigm shift in the way information is published and retrieved on a network, its design, shown at a high level in the right-hand image of Figure \ref{fig:hourglass}, borrows the same hourglass design as IP networks. Observe that upper layers of the network stack still promote the development of robust applications based on the underlying communication layers. The difference, however, is that network traffic flow management (i.e., to enable reliable and stable communication) and security are \emph{built into} the network stack. These architectural differences mean that application, transport, and network layer protocol semantics in IP-based networks are distinct from protocol semantics in NDN networks. The NDN gateway is intended to bridge between IP and NDN networks by performing this semantic translation between protocols. 

\begin{figure}
\begin{center}
\includegraphics[scale=0.5]{./images/hourglass_conn.pdf}
\label{fig:hourglass}
\caption{TODO}
\end{center}
\end{figure}

The gateway middleware is designed so as to support bi-directional traffic flowing from both types of networks. In what follows we describe how traffic in both directions will be supported internally by the gateway.

\begin{table}[t]
    \begin{tabular}{|c||c|c|}
    \hline
    ~    & {\bf IP} & {\bf NDN} \\ \hline
    {\bf HTTP} & ~ & ~ \\ \hline
    {\bf FTP}  & ~ & ~ \\ \hline
    {\bf SMTP} & ~ & ~ \\ \hline
    {\bf DNS}  & ~ & ~ \\ \hline
    \end{tabular}
\end{table}

\subsection{IP-to-NDN Traffic}
TODO

\subsection{NDN-to-IP Traffic}
TODO
\section{Implementation Overview}
\sink\ is implemented entirely in Python using the PyCCN \cite{pyccn} wrapper around CCNx. Its design can be viewed from two perspective: the gateway component and the bridge component. Though the bridge component uses elements of the gateway to minimize redundant code, the designs of each are mostly independent. In this section we describe each of the respective designs in more detail. 

\subsection{Gateway Design}
The core design of the gateway can be perceived as the composition of two flexible pipelines that route traffic from NDN (resp. IP) networks to IP (resp. NDN) networks (see Figure \ref{fig:pipeline}). Each pipeline begins and ends with an {\tt InputStage} and {\tt OutputStage}, respectively, one for the IP network and one for the NDN network. Each {\tt InputStage} instance runs an appropriate interface to the network from which it receives traffic. Specifically, the {\tt IPInputStage} runs an HTTP server to intercept IP-to-NDN interests, and the {\tt NDNInputStage} registers a CCNx handle and configures an interest filter for incoming interests. 

\begin{figure*}[ht!]
\begin{center}
\includegraphics[scale=0.45]{./images/pipeline.pdf}
\label{fig:pipeline}
\caption{Bidirectional message pipeline for IP-to-NDN and NDN-to-IP message traversal.}
\end{center}
\end{figure*}

It is important to emphasize that the {\tt NDNInputStage} operates independently of the underlying network implementation. In a true deployment, CCNx would not be used to interface with the NDN network. Rather, a CCN network stack would be implemented on top of the NDN network interface controller. The mechanism for registering an interest filter on top of this network stack would change, but the functionality that happens after an interest is intercepted will remain the same. Using CCnx for the preliminary development of this gateway was necessary since there does not yet exist NDN NICs or CCN software stacks that are not built upon the TCP/IP stack.

After an input stage receives an incoming message (i.e., an IP packet or NDN interest), the asynchronous message handler will allocate an entry in the \emph{pending message table} for the network interface, decompose the components of the message into its ``raw form'' and save them in common message object wrapper, and forward the resulting object to the next pipeline stage. For example, the {\tt IPInputStage} will save the IP source host and port information in the {\tt IPPendingMessageTable}, extract the URI path and set it as the ``destination'' field in the outgoing message object, and forward this object to the next pipeline stage. In addition to the message source information for each entry in the pending message table, a binary semaphore is also stored. After forwarding the object, each handler will acquire a lock on this binary semaphore until the content associated with this message is retrieved from the target network. 

The pipeline is designed so that each stage (with the exception of the {\tt InputStage}) has a thread-safe input queue and a reference to the next stage (with the exception of the {\tt OutputStage}). This simple interface enables any number of intermediate stages to be configured between the input and output stages. For simplicity, the \sink\ implementation only uses two stages - input and output stages. 

Once a message reaches the output stage a new message for the target network is created and issued to the network. For example, an NDN interest will be formatted based on the contents of the outgoing message object retrieved from the output stage queue. 

After the network responds with content, the input stage of the target network performs the following tasks. First, the pending message table is checked for the an entry corresponding to the input message (e.g., an interest name that matches name of a previously issued interest). If an entry is found, the content field of the matching entry is populated and the binary semaphore associated with the semaphore is released. The latter step unblocks the original asynchronous input stage message handler, which then retrieves the content from the pending message table entry and sends it to the original consumer. If a matching entry in the pending message table is not found, the incoming message is treated as a ``new'' message, and it is formatted and flows through the pipeline in the opposite direction. 

% \todo[inline]{pipeline stage defers parsing to subclasses - template pattern}

\subsection{Bridge Design}
As illustrated by Figure \ref{fig:pipeline}, the bridge component of \sink\ is designed to interoperate with the {\tt NDNInputStage} and {\tt NDNOutputStage} stages. 
\section{Performance Evaluation}

- test machine setup
- experimental procedures and applications
- link to source code
- message translation overhead both ways
- unidirectional and bidirectional RTT
- bridge latency
- say key generation and directory updates are asynchronous and one-time (don't happen a lot, so we didn't measure them)


\begin{figure}
\begin{center}
\includegraphics[scale=0.3]{./images/small.png}
\end{center}
\end{figure}

\begin{figure}
\begin{center}
\includegraphics[scale=0.3]{./images/large.png}
\end{center}
\end{figure}



\section{Conclusion}
TODO

% \section{Acknowledgments}

\bibliographystyle{abbrv}
\bibliography{ref}  % sigproc.bib is the name of the Bibliography in this case


\end{document}


