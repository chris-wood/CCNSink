\section{NDN/CCNx Overview}
NDN is one of the three currently active NSF FIA (Future Internet Architecture) projects. CCNx is a closely related and more commercially-oriented design backed by PARC. The defining characteristic of both designs is that they decouple location of content from its original publisher. To obtain content, a consumer issues a request (called an {\em interest}) referencing the name of the desired content. An interest is routed, based on the specified name, rather than a destination address, over a sequence of routers, each of which keeps state of the forwarded interest; see below. Requested content might be found either in a router that cached it based on a prior interest, or at the producer. Regardless, each router is expected (though not mandated) to cache each content it forwards. In essence, router caches and addressable content enable NDN/CCNx to reduce congestion and latency by keeping content closer to consumers. 

An \emph{interest}, though intended to carry a meaningful (human-readable) URI-like name, can in fact carry arbitrary strings corresponding to any type of data, such as encoded binary. Upon receiving an interest, a router looks up the content by name in its \emph{content store} (CS), i.e., a cache. A match in the CS causes the associated content to be forwarded downstream over the same interface upon which the interest arrived and no state is kept. Interests that do not match any cached content are stored in a \emph{Pending Interest Table} (PIT) together with the incoming and the outgoing interfaces. The interest is forwarded based on the longest-prefix match in the local \emph{Forwarding Information Base} (FIB) table. Multiple interests matching the same name are collapsed into a single PIT entry to prevent redundant interests being sent upstream. Once a content matching a PIT entry is received by a router, it is cached (unless the interest explicitly asks not to cache this content) and forwarded to all incoming interfaces associated with the PIT entry. Finally, the PIT entry is flushed.

Beyond the pull-model that guarantees symmetric interest and content flow, content-centric traffic in NDN and CCNx has strong security implications. {\bf First}, security is tied to content, as opposed to the channel through which it flows. All sensitive content must therefore be encrypted to ensure confidentiality. Content integrity and origin authentication are guaranteed by mandatory digital signatures; all content producers are required to sign content before responding to interests. Due to performance considerations, content signature verification, though possible, is not required by routers. Whereas, consumers are expected to verify all content signatures. Issues regarding signature verification and public key as well as trust management are discussed at length in \cite{ghali2014elements}. {\bf Second}, a lack of source and destination addresses in interest and content packets benefits privacy and anonymity. However, as discussed in the following section, this is insufficient to attain strong anonymity. 