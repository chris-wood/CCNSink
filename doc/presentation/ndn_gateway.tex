\documentclass[handout]{beamer}
\usetheme{Montpellier}
% \usecolortheme{beaver}

\usepackage{beamerthemesplit}
\usepackage{pgfpages}
\usepackage{verbatim}
\usepackage{fancybox}
\usepackage{algorithm}
\usepackage{amsmath}
\usepackage{amsthm}
\usepackage{algpseudocode}
\usepackage{algorithmicx}% http://ctan.org/pkg/algorithmicx
\usepackage{lipsum}% http://ctan.org/pkg/lipsum
\usepackage{xifthen}% http://ctan.org/pkg/xifthen
\usepackage{needspace}% http://ctan.org/pkg/needspace
\usepackage{hyperref}% http://ctan.org/pkg/hyperref
\usepackage{tikz}
\usepackage{mathptmx}
\usepackage[scaled=.90]{helvet}
\usepackage[T1]{fontenc}
\usepackage{framed}
\usepackage{listings}



\title[NDN Gateway]{Multipurpose IP/NDN Gateway and Bridge \\ for Heterogeneous Network Interoperability}
\institute[Donald Bren School of Information and Computer Sciences \\ UC Irvine]{}
\date{\today}
%\subtitle{}
\author[Christopher A. Wood]{Christopher A. Wood \\ \url{www.christopher-wood.com} \\ {\tt woodc1@uci.edu}}
%\institute[]{}
\date{\today}

\begin{document}

%%%%%
%%
%% Resource link: http://www.math-linux.com/spip.php?article77
%%
%%%%

\begin{frame}
	\titlepage
\end{frame}

\begin{frame}{Agenda}
	\tableofcontents
\end{frame}

%% 6-8 slides each

\section{Overview}
\begin{frame}{Today's Internet: Communication Networks as Distribution Networks}
	The communication-centric design enables point-to-point communcation between any two parties:
	\begin{itemize}
		\item Names)boxes)and)interfaces)
		\item Supports)end5to5end)conversaAons)
		\item Provides)unreliable)packet)delivery)via)IP)datagrams)
		\item Compensates)for)simplicity)of)IP)via)complexity)of)TCP)
	\end{itemize}

	Important observations:
	\begin{itemize}
		\item Helped)facilitate)today’s)content5centric)world)butwas)never)designed)for)it)
		\item Fundamental)communicaAon)model:)point5to5point)conversaAon)between)two)hosts)(IP)interfaces)
		\item The)central)abstracAon)is)a)host)idenAfier)corresponding)to)an)IP)address)
	\end{itemize}
\end{frame}

\begin{frame}{NDN Architecture Overview}
	{\bf Motivating Observation}: What matters is \emph{content}, not \emph{where} it came from
	TODO: differences

	TODO: names/roles/objects
\end{frame}

\begin{frame}{Content Retrieval Strategies}
	TODO
\end{frame}

\section{Design}
\begin{frame}{Semantic Translations}
	TODO
\end{frame}

\begin{frame}{IP-to-NDN Traffic}
	TODO
\end{frame}

\begin{frame}{NDN-to-IP Traffic}
	TODO
\end{frame}

\begin{frame}{Pipeline-Based Load Balancing Design}
	TODO
\end{frame}

\section{Experimental Setups}
\begin{frame}{Experiments}
	TODO: describe the experimental setup -> unidirectional in both settings, bidirectional applications
\end{frame}

%% EXAMPLE REFERENCES 
% \begin{frame}
% 	\frametitle{References}
% 	All images taken from Google Developers documentation: 
% 	\begin{center}
% 		\url{https://developers.google.com/appengine/features/}
% 	\end{center}
% 	% \begin{thebibliography}
% 	% \bibitem CHANGE ME PLEASE
% 	% \end{thebibliography}
% \end{frame}

%% EXAMPLE FIGURE
% \begin{comment}
% \begin{figure}
% \centering
% \includegraphics[scale = 0.6]{images/sub_layer.jpg}
% \end{figure}
% \end{comment}

\end{document}
