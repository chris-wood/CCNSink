\documentclass[handout]{beamer}
\usetheme{Montpellier}
% \usecolortheme{beaver}

\usepackage{beamerthemesplit}
\usepackage{pgfpages}
\usepackage{verbatim}
\usepackage{fancybox}
\usepackage{algorithm}
\usepackage{amsmath}
\usepackage{amsthm}
\usepackage{algpseudocode}
\usepackage{algorithmicx}% http://ctan.org/pkg/algorithmicx
\usepackage{lipsum}% http://ctan.org/pkg/lipsum
\usepackage{xifthen}% http://ctan.org/pkg/xifthen
\usepackage{needspace}% http://ctan.org/pkg/needspace
\usepackage{hyperref}% http://ctan.org/pkg/hyperref
\usepackage{tikz}
\usepackage{mathptmx}
\usepackage[scaled=.90]{helvet}
\usepackage[T1]{fontenc}
\usepackage{framed}
\usepackage{listings}
\usepackage{mdframed}
\input{bnf}


\title[{\sf CCNSink}]{{\sf CCNSink}: Application-Layer Middleware \\ for TCP/IP and Content Centric Network Interoperability}
\institute[Donald Bren School of Information and Computer Sciences \\ UC Irvine]{}
\date{\today}
%\subtitle{}
\author[Christopher A. Wood]{Christopher A. Wood \\ \url{www.christopher-wood.com} \\ {\tt woodc1@uci.edu}}
%\institute[]{}
\date{\today}

\begin{document}

%%%%%
%%
%% Resource link: http://www.math-linux.com/spip.php?article77
%%
%%%%

\begin{frame}
	\titlepage
\end{frame}

\begin{frame}{Agenda}
	\tableofcontents
\end{frame}

\section{Overview}
\begin{frame}{Today's Internet: Communication Networks as Distribution Networks}
	The communication-centric design enables point-to-point communcation between any two parties:
	\begin{itemize}
		\item Names and interfaces
		\item Supports end-to-end conversations
		\item Provides unreliable packet delivery via IP datagrams
		\item Compensates for simplicity of IP via complexity of TCP
	\end{itemize}

	Important observation: Helped facilitate today's concent-centric world, \emph{but was never designed for it!}

	\medskip

	\emph{NDN is a new architecture designed for content-centric networking}
\end{frame}

\begin{frame}{Data vs Communication Networks}
	Distribution/data (DN) and communication (CN) networks differ in several key ways:
	\begin{table}
		\begin{tabular}{c|c|c} \hline
		~ & CN & DN \\ \hline
		Naming & Endpoints & Content \\
		Memory & Invisible \& limited & Explicit (storage = wires) \\
		Security & Communication process/channel & Content \\ \hline
		\end{tabular}
	\end{table}
\end{frame}

\begin{frame}{NDN Overview}
	Content-centric networking flips around the host-based model of the Internet architecture
	\begin{itemize}
		\item \emph{Content names}, rather than content locations, become addressable. 
		\item Content is retrieved via \emph{interests}, which are similar to URLs: 
		\begin{center}
			{\tt ccnx://rit/gccis/cs/spr/ramsey\_survey}
		\end{center}
		\item The network is permitted to store (cache) content that is in high demand
		\item End result: less traffic to/from the content's original source, better usage of network resources, less latency, etc etc.
	\end{itemize}
\end{frame}

\begin{frame}{NDN Overview (continued)}
	How is data actually retrieved? 
	\begin{itemize}
		\item A consumer $C$ sends out an \emph{interest} for content they desire.
		\item A router $R_i$ use the information in their forwarding information base (FIB) table and data in cached in their content store (CS) to handle incoming interests:
		\begin{enumerate}
			\item If content with the same name matches what's stored in the CS, return that content
			\item Else, store the interest in their pending interest table (PIT) (including the downstream router $R_{i-1}$ or consumer $C$ that made the request), and forward the request upstream to the next router $R_{i+1}$ based on their FIB.
			\item FIBs are configured using protocol similar to OSPF
		\end{enumerate}
		\item Once the interest is satisfied in $R_i$, the PIT entry is cleared, the content is cached, and the data is sent downstream to $C$ or $R_{i-1}$. 
 	\end{itemize}
\end{frame}

\begin{frame}{Interest Format}
	\begin{itemize}
		\item Interests are similar to URLs: 
		\begin{center}
			{\tt ccnx://rit/gccis/cs/spr/ramsey\_survey}
		\end{center}
		\item The {\tt /} character is a delimeter that separates name \emph{components}
		\item A component can be \emph{anything}, including binary data (e.g. ciphertext)
		\item Interests are matched to providers in FIBs using a standard longest-prefix rule (to my knowledge, interests in CSs must match completely)
	\end{itemize}
\end{frame}

% TODO: images of this happening 
\begin{frame}
	\frametitle{NDN in Action - \#1}
	\begin{figure}[h]
		\includegraphics[scale=0.4]{img/ccn_img1.pdf}
	\end{figure}
\end{frame}

\begin{frame}
	\frametitle{NDN in Action - \#2}
	\begin{figure}[h]
		\includegraphics[scale=0.4]{img/ccn_img2.pdf}
	\end{figure}
\end{frame}

\begin{frame}
	\frametitle{NDN in Action - \#3}
	\begin{figure}[h]
		\includegraphics[scale=0.4]{img/ccn_img3.pdf}
	\end{figure}
\end{frame}

\begin{frame}
	\frametitle{NDN in Action - \#4}
	\begin{figure}[h]
		\includegraphics[scale=0.4]{img/ccn_img4.pdf}
	\end{figure}
\end{frame}

\begin{frame}{NDN at a Larger Scale}
	\begin{figure}[h]
		\includegraphics[scale=0.3]{img/ndn_dist.pdf}
	\end{figure}
\end{frame}

\begin{frame}{Underlying Network Differences}
	How will similar applications on both networks communicate with vastly different network stacks?
	\begin{figure}[h]
		\includegraphics[scale=0.32]{./img/hourglass_conn.pdf}
	\end{figure}
\end{frame}

\begin{frame}{Motivation for {\sf CCNSink}}
	\textbf{Question}: If adopted, how will NDN be deployed?
	\begin{enumerate}	
		\item ``Turn off'' the Internet, swap in new hardware, and then flip the switch again
		\begin{itemize}
			\item Bad idea...
		\end{itemize}
		\item Incrementally ``roll out'' NDN hardware and slowly make it interoperable with existing IP network
		\begin{itemize}
			\item How to enable NDN-based applications to communicate with IP-based applications (and vice versa)?
			\item ...and how to do this without re-writing the transport/network layer of IP-based applications to use CCNx (i.e., implement NDN functionality on top of IP)?
		\end{itemize}
	\end{enumerate}
\end{frame}

\begin{frame}{Motivation for {\sf CCNSink}}
	\textbf{Question}: If adopted, how will NDN be deployed?

	\medskip

	{\bf Answer} (in other words, what {\sf CCNSink} does): 
	\begin{itemize}
		\item Use NDN-network edge gateways to hide the details of NDN/IP communication mechanics
		\item Translate IP messages to compliant NDN interests (and vice versa)
		\item Use NDN-network edge bridges to connect isolated NDN ``islands''
	\end{itemize}
\end{frame}

\section{Modes of Operation}
\subsection{Gateway Functionality}
\begin{frame}{Gateway Semantic Translations}
	\begin{figure}[h]
		\includegraphics[scale=0.4]{img/gateway_highlevel.pdf}
	\end{figure}
\end{frame}

\begin{frame}{IP-to-NDN Traffic}
	\begin{itemize}
		\item HTTP GET requests issued to get content with a similar name
		\begin{itemize}
			\item e.g., {\tt GET X.X.X.X:80/ndn/ccnx/name/of/content}
			\item The request path is mapped to the outgoing interest name 
		\end{itemize}
		\item TCP connections established to stream data to NDN producers
		\begin{itemize}
			\item Socket connection between IP-based client and gateway established, NDN producer name first sent, and then all remaining data is streamed
			\item The gateway partitions data from the socket and packs it into an interest for the desired NDN producer
		\end{itemize}
	\end{itemize}
\end{frame}

\begin{frame}{NDN-to-IP Traffic}

Interests are encoded according to a special grammar to enable the gateway to parse interests and issue them using the appropriate IP-based protocol

\begin{figure}
\begin{mdframed}
\begingrammar

\noindent

<ip-interest>:	'/$\dots$/ip/'<protocol>.

<protocol>:	'http/'<http-cmd>[{'/'<http-path>}] | 'tcp/'<tcp-ident>'/'<uri-encoded-string>. 

<http-cmd>: 'GET' | 'PUT' | 'POST' | 'DELETE'.

% <ftp-cmd>: 'ascii' | 'binary' | 'bye' | 'cd' | 'close' | 'delete' | 'get' | 'help' | 'lcd' | 'ls' | 'mkdir' | 'mget' | 'open' | 'put' | 'pwd' | 'quit' | 'rmdir'.

<http-path>: <uri> | <ip-address>[port]['/'<uri-encoded-string>]

<tcp-ident>: <SHA256-hash>'/'<nonce>. % nonce is the random ID associated with the TCP socket for constant-time lookup when going from ndn-to-ip, this must be set during the connection establishment phase from ip-to-ndn (if it doesn't exist in the TCP state table, then the NDN-side is creating the connection in one shot)

% <tcp-param>: <uri-encoded-string>.

% <uri-encoded-string>: TODO

% <number>:	<real-number>;
% 		"$\{$" <real-number> "," <real-number> "$\}$";
% 		{$\backslash$}b[01][01]+;
% 		{$\backslash$}o[07][07]+;
% 		\$[0-9A-Fa-f][0-9A-Fa-f]+.

%<real-number>:	[\+--]?[0-9][0-9]+[\.[0-9]+]?[[eE][0-9][0-9]+].

% <operator>:	"*" |	 "/"	|     "$\backslash$"	| "\%";
% 		"==" |	 "!="	|     "$>$" 		| "$<$"  
% 		| "$<$=" | "$>$=";
% 		"\ul ="	| "\ul !=" |  "\ul $<$" | "\ul $>$" 
% 		| "\ul$<$=" | "\ul$>$=";
% 		"\&"	 | "$\vert$"  | "$\uparrow\uparrow$";
% 		"\&\&"	| "$\Vert$"  | "\ul$\uparrow$".
		
\endgrammar
\end{mdframed}
\end{figure}

\end{frame}

\subsection{Bridge Functionality}
\begin{frame}{Bridging NDN Networks}
	{\sf CCNSink} is used to bridge interests and their corresponding content across physically disjoint NDN networks.

	\medskip

	... But the devil is in the details
	\begin{itemize}
		\item NDN stipulates that all content is signed by its producers
		\begin{itemize}
			\item Content must be signed and verified as it crosses between two bridges, and then \emph{re-signed} before sent to the intended NDN consumer
		\end{itemize}
		\item Bridging should incur minimal overhead and handle high loads
		\begin{itemize}
			\item Use keyed MAC algorithms (e.g., HMAC) to tag and verify content as it traverses bridges (instead of digital signatures)
		\end{itemize}
		\item Bridges must be able to locate and connect to other bridges
		\begin{itemize}
			\item Use a central directory service to maintain knowledge of all bridges (updated periodically via heartbeat messages)
		\end{itemize}
	\end{itemize}
\end{frame}

\begin{frame}{The Bridge}
	Bridges establish persistent TCP connections to stream interests and content between disjoint networks. 
	\begin{figure}[h]
		\includegraphics[scale=0.5]{img/island_tunnel.pdf}
	\end{figure}
\end{frame}

\begin{frame}{The Bridge Directory (and the Director)}
	A central directory (to help establish pair-wise keys) or directory (to help establish a shared group key) can be used to manage the bridges and their keys.
	\begin{figure}[h]
		\includegraphics[scale=0.5]{img/gateway_group.pdf}
	\end{figure}
\end{frame}

\section{Internal Design and Implementation}
\begin{frame}{Pipeline-Based Load Balancing Design}
	\begin{figure}[h]
		\includegraphics[scale=0.34]{img/pipeline.pdf}
	\end{figure}
\end{frame}

\begin{frame}{Design Highlights}
	\begin{enumerate}
		\item Simple pipeline stage interface:
		\begin{itemize}
			\item Thread-safe input buffer queue of {\tt OutgoingMessage} objects
			\item Reference to ``next'' stage
		\end{itemize}
		\item Simple (and extensible) NDN-to-IP protocol multipliexing encoding
		\item Intuitive IP-to-NDN encoding method via HTTP GET requests
		\item All (IP and NDN) messages are handled asynchronously
		\begin{itemize}
			\item But synchronization primitives (semaphores/events) are used to wait for message responses
		\end{itemize}
	\end{enumerate}	
\end{frame}

\begin{frame}{Implementation Highlights}
	\begin{itemize}
		\item {\sf CCNSink} gateway/bridge: 
		\begin{itemize}
			\item Multi-threaded application written entirely in Python 
			\begin{itemize}
				\item Uses native Python libraries for thread synchronization primitives, IP-based communication (e.g., httplib)
			\end{itemize}
			\item Uses CCNx 0.82 for NDN communication
		\end{itemize}
		\item {\sf CCNSink} bridge directory: 
		\begin{itemize}
			\item Written entirely in Python
			\item Uses Python Flask library to communicate with bridge clients using HTTP
		\end{itemize}
	\end{itemize}
\end{frame}

\section{Experiments}
\begin{frame}{Performance: Experiments and Metrics}
	We assessed the design and implementation performance with the following experiments:
	\begin{itemize}
		\item Bidirectional ``application-layer'' and ``transport-layer'' communication across the gateway
		\item Unidirectional messages sent from IP and NDN hosts
	\end{itemize}
	We collected the following metrics:
	\begin{itemize}
		\item Unidirectional message translation overhead
		\item Unidirectional message trip time (RTT)
		\item Bridge mode message latency (RTT)
		\item Bridge mode symmetric key establishment overhead time
	\end{itemize}
\end{frame}

\begin{frame}{Relevant Overhead}
	\begin{itemize}
		\item IP-to-NDN translation: $\approx$0.00078s for interest names composed of one (1) to five (5) components
		\item NDN-to-IP translation: $\approx$0.0005s.
		\item Key agreement overhead: $\approx$0.186s.
	\end{itemize}
\end{frame}

\begin{frame}{RTT Results}
	\begin{figure}
	\begin{center}
	\includegraphics[scale=0.4]{./img/small.png}
	\label{fig:perf1}
	\end{center}
	\end{figure}
	Average RTT times for IP-to-NDN, NDN-to-IP, and bridge messages when requesting content of approximately 1MB in size.
\end{frame}

\begin{frame}{RTT Results (cont'd)}
	\begin{figure}
	\begin{center}
	\includegraphics[scale=0.4]{./img/large.png}
	\label{fig:perf2}
	\end{center}
	\end{figure}
	Average RTT times for IP-to-NDN, NDN-to-IP, and bridge messages when requesting content of 10MB in size.
\end{frame}

\begin{frame}{Bridge RTT Results for 100MB Content}
	\begin{figure}
	\begin{center}
	\includegraphics[scale=0.3]{./img/huge.png}
	\label{fig:perf3}
	\end{center}
	\end{figure}
	Average RTT time for bridge messages when requesting content of 100MB in size.
\end{frame}

\section{Future Work}
\begin{frame}{Future Work}
	Project paper submitted to ACM ICN 2014.

	\medskip

	Next tasks include:
	\begin{itemize}
		\item Implement group-based key establishment routine for bridges
		\item Expand NDN-to-IP encoding grammar to support more TCP/IP protocols
		\item Test current {\sf CCNSink} implementation under heavy message load with geographically distributed hosts
		\item Test {\sf CCNSink} on top of actual NDN hardware (i.e., not using CCNx as NDN interface)
	\end{itemize}
\end{frame}

%% EXAMPLE REFERENCES 
% \begin{frame}
% 	\frametitle{References}
% 	All images taken from Google Developers documentation: 
% 	\begin{center}
% 		\url{https://developers.google.com/appengine/features/}
% 	\end{center}
% 	% \begin{thebibliography}
% 	% \bibitem CHANGE ME PLEASE
% 	% \end{thebibliography}
% \end{frame}

%% EXAMPLE FIGURE
% \begin{comment}
% \begin{figure}
% \centering
% \includegraphics[scale = 0.6]{images/sub_layer.jpg}
% \end{figure}
% \end{comment}

\end{document}
