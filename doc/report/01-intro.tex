\section{Introduction}
At its core, the design and architecture of today's Internet is a communication-based packet switching network. Since its inception, it has been retrofitted with a variety of transport and application layer protocols and middleware to support a growing set of consumer applications, such as the Web, email, and perhaps most importantly in recent times, media streaming services. The latter type of applications are bandwidth-intensive content distribution entities which leverage the underlying communication-based network as a distribution network, leading to a massive consumption of vital, and sometimes scarce, networking resources. 

Content-centric networks (CCNs) are a new class of network architecture designs that aim to address this increasingly popular class of applications by decoupling data from its source and shifting the emphasis of addressable hosts and interfaces to content \cite{first}. By directly addressing content instead of hosts, content dissemination and security can be ``distributed'' throughout the network in the sense that consumers requests for content are satisfied by \emph{any} resource in the network (i.e., not necessarily the original producer). For example, routers close to consumers may cache content with a particular name and then satisfy all content requests that match the content's name. In-network caching and data-centric security measures are two of the defining characteristics of these new network designs. 

As of today there are several content-centric networking proposals being explored as alternative designs to today's Internet; Named Data Networking (NDN) \cite{ndn-techreport} is one of the more promising designs that is still an active area of active research (see \url{www.named-data.net} for more information). As a replacement for IP-based networks, the complete adoption of NDN, or any one of these designs, will realistically need to be done by slow and continual integration and replacement of IP-based networking resources with NDN-based resources. Currently, however, there is no engineering plan to support the IP/CCN integration without significant software modification and application, transport, or network layer source code modifications (e.g., integrating and using CCNx to communicate with CCN-based applications from TCP/IP hosts).

Consequently, the primary objective of \sink\ is to aid the integration of future content-centric networking resources into the existing IP-centric Internet by providing a middleware to support IP and NDN interoperability\footnote{We assume that NDN \emph{will not} be deployed over the IP network, as such a deployment scheme would nullify the need for this type of middleware between the two networks.}. Application-layer traffic corresponding to protocols such as HTTP will be translated by middleware to correctly interface with NDN resources, thereby serving as a semantic gateway between these two fundamentally different networking architectures. Similarly, using a custom NDN-to-IP interest naming convention, NDN \emph{interests} for content will be translated to messages adhering to application-layer protocols to traverse the IP network. \sink\ also serves to bridge isolated NDN resources. In this use case, two \sink\ bridges will leverage the features of the IP network to forward interests and the respective content across separated NDN ``islands.'' The primary benefit of this semi-transparent middleware is that existing IP-centric applications need not be modified at any layer in the network stack to interoperate with NDN resources. Furthermore, NDN-based applications can communicate across physically partitioned networks so long as there exists \sink\ bridges between them. 

The rest of this paper is outlined as follows. Section 2 provides an overview of NDN and CCNx as they pertain to this work, and Section 3 highlights some of the motivations for the gateway and bridge. The design of \sink\ is presented in Section 4, followed by implementation and performance details in Section 5 and 6, respectively. Finally, we conclude with a discussion of unfinished work and avenues for further development in Section 7. 

% \todo[inline]{Emphasize notlike a broker}

% \todo[inline]{we assume NDN is NOT deployed over IP-i.e., computers may speak IP or NDN, but not both}

% \todo[inline]{The rest of this paper is outlined as follows...}


