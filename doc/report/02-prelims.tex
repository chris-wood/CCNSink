\section{NDN/CCNx Overview}
NDN is one of the three currently active NSF FIA (Future Internet Architecture) projects \cite{ndn-techreport}. CCNx is a closely related and more commercially-oriented design backed by PARC \cite{ccnx}\footnote{Due to the similarity between NDN and CCN with respect to this work, we use the terms NDN and CCN interchangeably}. The defining characteristic of both designs is that they decouple location of content from its original publisher. To obtain content, a consumer issues a request (called an {\em interest}) referencing the name of the desired content. An interest is routed based on the specified name, rather than a destination address, over a sequence of routers, each of which keeps state of the forwarded interest. Requested content might be found either in a router that cached it based on a prior interest, or at the producer. Regardless, each router is expected (though not mandated) to cache each content it forwards. In essence, router caches and addressable content enable NDN/CCNx to reduce congestion and latency by keeping content closer to consumers. 

An \emph{interest}, though intended to carry a meaningful (human-readable) URI-like name, can in fact carry arbitrary strings corresponding to any type of data, such as encoded binary. Upon receiving an interest, a router looks up the content by name in its local cache, deemed the \emph{content store} (CS). A match in the CS causes the associated content to be forwarded downstream over the same interface upon which the interest arrived. Interests that do not match any cached content are stored in a \emph{Pending Interest Table} (PIT) together with the incoming and the outgoing interfaces. The interest is forwarded based on the longest-prefix match in the local \emph{Forwarding Information Base} (FIB) table. Multiple interests matching the same name are collapsed into a single PIT entry to prevent redundant interests being sent upstream. Once a content matching a PIT entry is received by a router, it is cached (unless the interest explicitly asks not to cache this content) and forwarded to all incoming interfaces associated with the PIT entry. Finally, the PIT entry is flushed.

One may perceive the pull-based model of content retrieval as an analogue of traditional publish-subscribe distributed systems \cite{eugster2003many}. In fact, the two designs are similar in the sense that interested parties subscribe to content and then receive new content when it becomes available. However, the key difference between these two systems is that NDN (and ICNs in general) ``subscriptions'' are ephemeral. A \emph{single} interest, which is analogous to a subscription, is issued for a \emph{single}, \emph{discrete} piece of content. In contrast to publish-subscribe systems, which push new content to all subscribed users when it becomes available, new NDN content will not be sent to consumers unless they issue another explicit interest in such content. Clearly, publish-subscribe systems can be used to implement NDN. However, this strategy would be extremely wasteful in terms of both computation and storage. Firstly, subscriptions would be continually added and removed for every interest, incuding a lot of computational overhead. Secondly, routing information is distributed in NDN routers via routing tables, whereas routing information would be isolated in clients or a global directory service used to locate content publishers. 

The interest-based strategy for content naming may also be contrived as a close sibling to MIT's Intentional Naming System (INS) \cite{INS}. In fact, the naming schemes and routing strategies are quite different. The INS uses name specifiers composed of attributes (keys) and values, e.g., {\tt [city = washington [building = whitehouse]] [[service = camera [data-type = picture]]] [name = president.jpg]}. The INS resolution service parses each component of these name specifiers to locate the source of the specified data (in this case, where the picture named ``president.jpg'' can be located). In NDN, such a picture might be retrieved by issuing an interest with the name {\tt ccnx:/washington/whitebouse/service/camera/picture/name/president.jpg}. In addition to these explicit naming differences, it is important to notice that INS uses distributed \emph{interest name resolvers} (INRs), similar to hierarchichal DNS servers, to direct name specifiers to content providers, whereas NDN interests are iteratively resolved by routers in the network using their internal routing tables. Active INRs, which effectively serve as ``routers'' of name specifiers, are managed and discovered using a DNS-like service, whereas actual NDN routers are discovered using standard routing protocols (e.g., OSPF, RIP, etc). Clearly, these subtle differences make NDN and INS differentiate to a significant degree. However, INS is in some sense more advanced than NDN in that it also provides a name discovery service. 

Beyond the pull-based content retrieval model that guarantees symmetric interest and content flow, content-centric traffic in NDN and CCNx has strong security implications. Most importantly, security is coupled to content rather than the channel of distribution. All sensitive content must therefore be encrypted in a meaningful way so as to ensure confidentiality. Content integrity and origin authenticity are ensured by mandating that all content be digitally signed by its producers. As we will discuss in the following sections, this requirement plays a crucial role in the bridge component of \sink. Contrary to generating digital signatures, routers need not verify signatures as content flows through the network due to the obvious computational overhead. On the contrary, consumers are assumed to verify all content signatures and re-issue interests in the event that signature verification fails. Issues regarding signature verification and public key distribution are elaborated upon in \cite{ghali2014elements}. 

